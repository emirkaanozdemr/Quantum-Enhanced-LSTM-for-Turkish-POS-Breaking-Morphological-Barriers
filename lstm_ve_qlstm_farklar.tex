\documentclass[a4paper,12pt]{article}
\usepackage{amsmath, amssymb, amsfonts}
\usepackage{geometry}
\geometry{margin=1in}
\usepackage{hyperref}

\begin{document}

\title{Klasik LSTM ve Quantum LSTM Karşılaştırması}
\author{}
\date{}
\maketitle

\section*{Giriş}
LSTM (Long Short-Term Memory), zaman serisi ve sıralı verilerdeki uzun vadeli bağımlılıkları modellemek için kullanılan bir yapay sinir ağıdır. Quantum LSTM (QLSTM), LSTM'nin quantum hesaplama paradigmalarına uyarlanmış bir versiyonudur. Bu belgede, her iki modelin temel matematiksel formülleri karşılaştırılacaktır.

\section*{Klasik LSTM Formülleri}
Klasik LSTM, aşağıdaki kapılardan oluşur:

\begin{itemize}
    \item \textbf{Unutma Kapısı:}
    \begin{equation}
    f_t = \sigma(W_f h_{t-1} + U_f x_t + b_f)
    \end{equation}
    \item \textbf{Giriş Kapısı:}
    \begin{equation}
    i_t = \sigma(W_i h_{t-1} + U_i x_t + b_i)
    \end{equation}
    \item \textbf{Aday Hücre Durumu:}
    \begin{equation}
    \tilde{C}_t = \tanh(W_C h_{t-1} + U_C x_t + b_C)
    \end{equation}
    \item \textbf{Hücre Durumu Güncellemesi:}
    \begin{equation}
    C_t = f_t \odot C_{t-1} + i_t \odot \tilde{C}_t
    \end{equation}
    \item \textbf{Çıkış Kapısı:}
    \begin{equation}
    o_t = \sigma(W_o h_{t-1} + U_o x_t + b_o)
    \end{equation}
    \item \textbf{Gizli Durum:}
    \begin{equation}
    h_t = o_t \odot \tanh(C_t)
    \end{equation}
\end{itemize}

Burada:
\begin{itemize}
    \item $x_t$: Giriş vektörü,
    \item $h_{t-1}$: Bir önceki gizli durum,
    \item $C_{t-1}$: Bir önceki hücre durumu,
    \item $W$, $U$: Ağırlık matrisleri,
    \item $b$: Bias terimleri,
    \item $\sigma$: Sigmoid aktivasyon fonksiyonu,
    \item $\odot$: Element bazlı çarpma.
\end{itemize}

\section*{Quantum LSTM Formülleri}
Quantum LSTM, klasik LSTM'nin quantum bilgi işlem paradigmalarına göre uyarlanmış bir versiyonudur. Bu modelde, klasik işlemler quantum durumları ve operatörleri ile gerçekleştirilir.

\begin{itemize}
    \item \textbf{Unutma Kapısı:}
    \begin{equation}
    f_t = \sigma(U_f \cdot |\psi_{t-1}\rangle + V_f \cdot |x_t\rangle + b_f)
    \end{equation}
    \item \textbf{Giriş Kapısı:}
    \begin{equation}
    i_t = \sigma(U_i \cdot |\psi_{t-1}\rangle + V_i \cdot |x_t\rangle + b_i)
    \end{equation}
    \item \textbf{Aday Hücre Durumu:}
    \begin{equation}
    \tilde{C}_t = \tanh(U_C \cdot |\psi_{t-1}\rangle + V_C \cdot |x_t\rangle + b_C)
    \end{equation}
    \item \textbf{Hücre Durumu Güncellemesi:}
    \begin{equation}
    C_t = f_t \odot C_{t-1} + i_t \odot \tilde{C}_t
    \end{equation}
    \item \textbf{Çıkış Kapısı:}
    \begin{equation}
    o_t = \sigma(U_o \cdot |\psi_{t-1}\rangle + V_o \cdot |x_t\rangle + b_o)
    \end{equation}
    \item \textbf{Gizli Durum:}
    \begin{equation}
    h_t = o_t \odot \tanh(C_t)
    \end{equation}
\end{itemize}

Burada:
\begin{itemize}
    \item $|\psi_{t-1}\rangle$: Bir önceki gizli durumun quantum temsili,
    \item $|x_t\rangle$: Quantum giriş vektörü,
    \item $U$, $V$: Quantum ağırlık operatörleri.
\end{itemize}

\section*{Klasik ve Quantum LSTM Karşılaştırması}
\begin{itemize}
    \item Klasik LSTM'de tüm hesaplamalar deterministik bir şekilde yapılırken, QLSTM süperpozisyon ve dolanıklık gibi quantum özellikleri kullanır.
    \item Quantum LSTM, parametre boyutunu azaltarak büyük ölçekli problemleri quantum paralellik ile çözebilir.
    \item Klasik LSTM'nin aktivasyon fonksiyonları, QLSTM'de quantum kapıları ile gerçekleştirilir.
\end{itemize}

\end{document}
